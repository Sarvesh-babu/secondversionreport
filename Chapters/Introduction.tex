	\chapter{INTRODUCTION}
	\label{chap:intro}
	
	
	With the increase in Pollution, fuel demand, global warming and many other socio-economic issues, one could say Electric Vehicles (EV) will be the future means of transport. Transportation sector can be tremendously changed by Electric Vehicles as there will be less carbon emissions.
	
	EVs help to conserve non-renewable natural resources. Due
	this the import of natural gasses and the dependency of a nation on petroleum export countries
	will be reduced. When compared to the recurring expenditure on natural gasses, the cost of
	EVs are low. The maintenance of electric motors is less when compared with traditional non-
	electric motors. Electric vehicles have different methods for conversion of fuel to electricity.
	
	
	
	\par Electric bikes, Cars, rickshaws, trucks, etc are some examples of EVs. Most of the trains including metros are already running through electricity all over the world, but there is a problem in charging EVs in comparison with the motor vehicles EVs can't be charged instantly like Petrol/Diesel Vehicles. 
	So here there is a need for an effective charging scheme.
	
	\par But in present times there are difficulties based on how to charge EV s when there is higher
	load demand. The problem arises when there are many vehicles at the same time to charge
	and prioritize them accordingly, there are lesser number of charging stations and the charging
	the time of vehicles differs for different vehicles. Replacement of battery packs can be one such
	solution but that is quite expensive for people to afford. There will be a significant impact on the power grid by charging EVs. In order to manage EVs charging, there is a need for an intelligent
	charging strategy that supports EVs charging while preventing the power grid from
	overloading. Charging/discharging pattern scheme using Real Time Price can be one possible
	solution for the difficulties mentioned above and the same is discussed below.
	
	\par Today when advancement of technology is growing rapidly EVs are the new means of transport to fulfill the larger demand of people growing day by day. One more important factor to consider regarding EVs is the cost of electricity. There are various types of tariffs followed by the government depending on various factors like maximum demand, Time of the load, Type of the load, amount of the energy used, etc. Real time price is one such system where the prices vary hourly and the consumer is charged a different price for each interval, here price solely depends on the demand in the network prevailing during the time. This benefits the consumers as well the government in cutting down their losses. Also, this system could make the load duration curve flatter.


	\par The remainder of the report is organized as follows:
	\begin{itemize}
		\item A literature survey related to the work and the formulated objectives are presented in the chapter \ref{chap:LR}.
		
		\item The chapter \ref{chap:methodology}  provides detailed explanation
		of methodology and the system model comprising of EV.
				
		\item The chapter \ref{chap:mathmodel} deals with the mathematical modelling of the systems used.
		
		\item The chapter  \ref{chap:results} consists of the results obtained for the considered test systems and are discussed and summarized. 
		\item The conclusions of the work is presented in the chapter \ref{chap:conclusion} with a brief description of the future scope of this work.
	\end{itemize}
