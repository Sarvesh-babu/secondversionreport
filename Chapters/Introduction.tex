	\chapter{INTRODUCTION}
	\label{chap:intro}
	
	
	With the increase in Pollution , fuel demand, global warming and many other socio-economic issues, one could say Electric Vehicles(EV) will be the future means of transport.

Pollution, demand for fuel, global warming are the major problems in our livelihood and EVs
are eco-friendly means of transport. This is why EVs are been promoted or has been brought
into the market in recent times.

These vehicles may be powered through the charging docs or a
collector system from off- vehicle sources or may have in-built battery system, solar panel,
fuel cell or electric generator to convert fuel to electricity to drive the vehicle. Electric bikes,
cars rickshaws, trucks, etc are some examples of EVs. Most of the trains including metros are
already running through electricity all over the world.

Many factors have made EVs an urgent need in the present. Electric vehicles have the potential to
reshape the transportation sector, drastically cutting carbon emissions and clearing the way
for significant climate progress. EV s help to conserve non-renewable natural resources. Due
this the import of natural gasses and the dependency of a nation on petroleum export countries
will be reduced. When compared to the recurring expenditure on natural gases, the cost of
EVs are low. The maintenance of electric motors is less when compared with traditional non-
electric motors. Electric vehicles have different methods for conversion of fuel to electricity.

One of the method is battery storage. To know about the battery storage, the vehicle’s battery
capacity must be known. Here concentrating on vehicle’s battery capacity it can range from
below hundred KWH to above hundred kWh for different vehicles.

But in present times there are difficulties based on how to charge EV s when there is higher
load demand. The problem arises when there are many vehicles at the same time to charge
and prioritize them accordingly, there are lesser number of charging stations and the charging
time of vehicles differ for different vehicles. Replacement of battery packs can be one such
solution but that is quite expensive for people to afford. Charging EVs will have a significant
impact on the power grid. In order to manage EVs charging, there is a need for an intelligent
charging strategy that supports EVs charging while preventing the power grid from
overloading. Charging/discharging pattern scheme using Real Time Price can be one possible
solution for the difficulties mentioned above and the same is discussed below.

Today when advancement of technology is growing rapidly EV s are the new means of transport to
fulfil the larger demand of people growing day by day.
\par The remainder of the thesis is organized as follows:
\begin{itemize}
	\item A literature review related to the work and the formulated objectives are presented in the chapter \ref{chap:LR}.
	
	\item The chapter \ref{chap:methodology} provides detailed explanation
	of methodology and the system model comprising of solar PV, wind-based renewable DG, and EV are presented in chapter \ref{chap:mathmodel}.
	\item The chapter \ref{chap:results} results obtained for the considered test systems are discussed and summarized. 
	\item The conclusions of the work is presented in the chapter \ref{chap:conclusion} with a brief description of the future scope of this work.
\end{itemize}
