	\chapter{Mathematical Modelling}
	\label{chap:mathmodel}
	
	\section{SoC Calculation}
	
	
		
	The SoC of the vehicle is calculated from the following equations:
	

			\begin{equation}
		                             SoC_{min} \leq SoC \leq SoC_{max}\label{eq:socminmax}
		    \end{equation}

		    \begin{equation}                         
		               SoC_{y} = SoC_{y-1} + P_{batt}(y) \times \partial y \times \eta c
			\end{equation}
		    \begin{equation}             
		               SoC_{y} = SoC_{y-1} - P_{batt}(y) \times \partial y \times \eta d
			\end{equation}
		    \begin{equation}             
		               P_{batt}(y) = SoC_{y} \times E
		    \end{equation}
		 
\begin{equation}             
	 Initial Power = Generation - Load
\end{equation}

	  SoC limits:  
	 
	   $SoC_{min}$ and $SoC_{max}$ are  the maximum and minimum SoC of the EV respectively.
	   This constraint allows the SoC to vary between predefined minimum and maximum SoC.               
	
	\section{Best Pattern for charging}
	
	
	The charging pattern is determined by comparing the Energy required to the Real Time Price and by  identifying the minimum of it. 
	
	\begin{center}
		
		$T_{n}$ = $\sum^{24}_{i=1}$ ($Ch_{t} $ $\times$ [1 || 0 || -1 ] ) $\ast $ $Rtp_{i}$ 
    \end{center}	 
	\section{Maximum Power required by EV}
	
	
	Maximum Power demand occurs when all the three vehicles loads are high and the time block of maximum demand is identified. 
	
	\begin{center}
		
		$P_{t(total)}$ = $P_{t(car)}$ + $P_{t(truck)}$ + $P_{t(bus)}$
		
		$P_{t(total)}$ = argmax $\pi_{i}^{24}$ $\ast$ $P_{t(total)}$
	\end{center} 
\section{Equation References - Siva}
\begin{equation}%\label{eq}
	\begin{split} 
		Z=\sum_{t=1}^{24}[P_{Fix}(t)*C(t)+\alpha*P_{Curt}(t)*C(t) &+\beta*P_{Contr}(t)*C(t)]  -  &[RTP(t)*P_{G}(t) + P_{RE}(t)] 
	\end{split} 
\end{equation}

where, `Z' is the profit of the utility, `$P_{Fix}(t)$' , `$P_{Curt}(t)$' and `$P_{Contr}(t)$' are the fixed, curtailable and controllable power demand at the time duration `t' in $kW$.
`$P_{G}$' and `$P_{RE}$' is the total power from grid and power generated by renewables at time `t' in $kW$, `$C(t)$' is cost at that specific time interval in $\$ $, RTP is real time price in $\$ $ and `$\alpha$' and `$\beta$' is potential of load curtailment and shifting at time `t' respectively. In this work, five different cases are investigated to analyze the impact of the proposed DR in improving the utility's profit. 
\par When the power generated by renewable resources is used to supply the required total demand with the remaining power required being supplied from the grid. Then total power is,
\begin{equation}
	P_{Total} =\sum_{t=1}^{24}[P_{RE}(t)+P_{G}(t)]
\end{equation}
\par The condition of load shifting is not required if the total available load is supplied by the renewables. This is stated as,
\begin{equation}
	P_{RE}(t)>P_{G}(t)
\end{equation}
i.e. Maximum  available power is greater than the required load. 
\begin{equation}
	P_{max}(t)\ge P_{load}(t)
\end{equation}                      
where, `$P_{max}(t$)' is the maximum avaialable power at a particular time `t' in $kW$ and `$P_{load}(t)$' is total power required by the grid in specific time interval `t' in $kW$. Whereas, the required energy is supplied by the renewables in assistance with grid can be termed as,
\begin{equation}
	P_{RE}(t) < P_{load}(t) < P_{max}(t)
\end{equation}       
Meanwhile when the necessary demand is met by the renewables and supply from the grid along with load shiting can be expressed as,
\begin{equation}
	P_{RE}(t) < P_{max}(t)< P_{load}(t) 
\end{equation}       
\begin{equation}						 {P_s(i)=\frac{\sum_{k=1}^{i}fitness (k)}{\sum_{j=1}^{N}fitness (j)}}
\end{equation}

AENS is average energy not supplied index and is expressed in terms of kilowatt hour per customer.
\begin{equation}
	{AENS=\frac{Total\,energy\,not\,supplied\,}{Total\, number\,of\,customers\,served}}
	\label{eq:AENS_eq}
\end{equation}
\begin{equation}
	AENS =  \dfrac{\sum\limits L_{a(i)}U_{i}}{N_{i}}
	\label{eq:AENS}
\end{equation}
\begin{equation}
	A=\frac{1}{{\left(V_{ci-in}-V_r\right)}^2}\left\{V_{ci-in}\left(V_{ci-in}+V_r\right)-4V_{ci-in}V_r\left[\frac{V_{c-in}+V_r}{2V_r}\right]}^3\right\} 
\end{equation}
\begin{equation}
B=\frac{1}{{\left(V_{c-in}-V_r\right)}^2}\left\{4\left(V_{c-in}+V_r\right){\left[\frac{V_{c-in}+V_r}{2V_r}\right]}^3-\left(3V_{c-in}+V_r\right)\right\} 
\end{equation}
\begin{equation}
C=\left(\frac{1}{{\left(V_{c-in}-V_r\right)}^2}\right)\left\{2-4{\left[\frac{V_{c-in}+V_r}{2V_r}\right]}^3\right\}
\end{equation}
`$P_{r}$' is the wind turbine's rated power in $kW$, `$V_{c-in}$' is wind turbine's cut-in speed in $m/s$, `$V_{c-out}$' is wind turbine's cut-out speed in $m/s$ and `$V_{r}$' rated speed of wind turbine in $m/s$.
\begin{equation}	
0 \leq  P_{w} \leq P_{w}^{max}
\end{equation}
Where `$P_{w}$' is the wind power generated in $kW$ and `$P_{w}^{max}$' is the  maximum available wind power in $kW$.

