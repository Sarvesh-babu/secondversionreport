\chapter{LITERATURE SURVEY}
	\label{chap:LR}
	
	\cite{base} This paper addresses the problem of stability with the grid .The load profile of electric vehicle charging stations (EVCS) is determined  and its impact of EVCS on the voltage profile of the distribution system. The principle of coordinated charging strategy is approached in this project to find the load profile of EVCS constrained by grid-to-vehicle (G2V) and vehicle-to-grid (V2G). 
	
	\cite{soc} This paper focuses on minimization of the losses in the distribution network by  the Distributed Generators (DGs) systems.The undesirable nature is one of the major challenges in maintaining the power between the DG and the grid. The battery energy storage system serves the purpose,supporting the grid by storing the energy in it. Sizing and placement of DG in the distribution system to support the grid and  battery storage placed after the placement of the DGs is done in this work.Particle Swarm Optimization (PSO) is used for this purpose and for placement of the battery energy storage system a genetic algorithm is used in the IEEE networks such as IEEE-33, IEEE-69, IEEE-118 radial node distribution systems.This project incorporates  the operation of the battery in accordance with the state of charge. 
	
	\cite{rtp} The paper focuses on Cost minimization and export maximisation  by optimising  the plug-in hybrid electric vehicle (PHEV) schedule , thereby reducing the import from the grid, subsequently minimising the overall operational cost. The proposed DR proposes to  reduce imported electricity cost in peak hours by shifting the non-emergency loads(controllable loads) to off-peak hours.In this project the per hour Real time pricing(RTP) and load profile has been incorporated and analysed.
	
	\cite{evdata} A dynamic EV charging dataset from the ACN data has been used as the reference to calculate the charging time, discharging time, kWh required, time required to full charge. With this data the charging-discharging pattern has been established. 
	
	\cite{33bus} This paper aims to establish a generalised procedure for evaluating the impact of Electric Vehicles Charging Stations (EVCS) on a distribution system. A load profile for 24 hours of EVCS is proposed using the travel patterns of EV .Radial Load Flow is performed on the distribution system.This is done when the EVCS is connected to the distribution system to find the voltage profile of the system. Voltage Stability Index (VSI) and Voltage Stability Factor (VSF) are calculated for the system with and without EVCS under many cases.
	
	\cite{rao} In this paper, an approach to reconfigure and install DG units simultaneously in the distribution system has been proposed which includes different loss reduction methods to establish the superiority of the proposed method. An effective meta heuristic HSA is used in the process of the network reconfiguration and installation of DG.33- and 69-bus systems at three different load levels vizLight, nominal, and heavy are used for testing the approached as well as other methods.The result was that the approached method was more effective in reducing power loss and improving the voltage profile as compared to other methods. Then this  was studied at different load levels. The results showed that the percentage power loss reduction was improving as the number of DG installation locations were increasing from one to four, but rate of improvement decreased when locations were increasing from one to four at all load levels. However,when the number of DG installation locations were three the ratio of percentage loss reduction to DG size was the highest. The HSA results were compared with the results of genetic algorithm (GA) and refined genetic algorithm (RGA). Final result was that the HSA performance was better than GA and RGA.
	
	\cite{69bus} In this paper, the consumer loss connected to a radial system has been examined by the Quadratic-loss allocation scheme, which is based on branch current flow. Therefore, it is ensured each consumer has been allocated the losses only at branches for which current contributes to.  A heuristic rule and fuzzy multiobjective algorithm in a radial distribution system is derived to solve the network reconfiguration problem. The analysis demonstrates that the network reconfiguration reduces system real-power loss and most consumers will pay less, due to reduction in the loss allocation. However, this may also result in an increase in the loss allocation to a small group of consumers. This can be addressed by modification of tariff structure. It is also observed that network reconfiguration would influence the real-power loss allocation to each consumer.
	
	\cite{zhang2022fast} The large-scale fast charging of EVs in the distribution network creates an issue of uneven temporal and spatial distribution of charging loads and voltage quality deterioration.  This can be addressed by adjustable charging service fees for adjustable charging service fees. The charging guidance strategy will guide the users to charge reasonably, helping to reduce the charging cost and the distribution network’s voltage quality is enhanced substantially.
	The spatial and temporal distribution is forecasted considering the grid constraints of urban traffic road network-distribution and fast charging loads of private cars based on trip chains and Monte Carlo methods is proposed. The load and voltage quality of the distribution network is taken into consideration and a flexible tiered charging service fee model is established.  The trip time, charging service cost and power consumption are considered to construct a user charging location.  The factors that are analysed for a weighted decision model are : voltage of distribution network nodes, user’s charging pattern,  the spatial and temporal distribution of regional charging loads. Other factors which are analysed include, revenue from charging station revenue \& impact on user charging cost under the proposed strategy. 
	The charging guidance strategy has impact on the distribution network, charging station operators and EV users,  apart from changing the spatial and temporal distribution of fast charging loads/
	The method results in the overall voltage deviation to 4.279 and significantly  improves the  voltage quality of the distribution network level. At user level, overall comprehensive charging cost is reduced to 9.28\% motivating to respond and adapt the strategy. 
	At charging station operators level, there has been a slight fall in their revenues \& profit by the adoption of the charging guidance strategy. However, it is envisaged that in future the profit of charging station operators would go up due to the V2G technology development and the increasing use of EV. 
	The approach of participation in grid interaction would be a futuristic research topic.
	
	
	\cite{simolin2022assessing} This paper assesses the influence on EV charging load through the charging profile modelling method.
	Realistic sampling with four commercial EVs are experimented and compared. Also, REDI shopping centre’s charging session data is used for a large charging site evaluation sample.
	It is found that, to avoid modelling errors, linear charging profile is not recommended for controlled charging.  Measurement-based nonlinear charging profiles wage would lead to most accurate modelling 
	However, usage of a simple, but justified, bilinear charging profile model is also recommended for a reasonably accurate result.
	
	Based on the calculations, it is derived that in the constant voltage stage, the charging currents of commercial EVs decrease around 15.3 mA/s per phase on an average. It is further observed that if this value is used to model charging profiles, it will lead to a reasonable low modelling error of 0.64\%–4.39\% for the highest hourly peak power and the charged energy. 
	Therefore, it would help to study EV charging load modelling from the charging site point-of-view, enhancing its accuracy with reduced computational requirements. Also, the paper demonstrates that in the charging site, the battery temperatures do not have a substantial influence on the charging loads and therefore, may not be considered for further related studies.
	The paper also derives that uncontrolled charging can be modelled more accurately when compared to  controlled charging. The different charging profile modelling  accuracy for different control algorithms needs to be investigated further. However, this can be a challenging task because of the limitations on the currently available commercial EVs. For instance, few EVs may not support vehicle-to-grid or  communicate information, such as SoC, to the control system. In addition, real time measurements are to be used as a baseline for the comparisons, otherwise different modelling methods cannot be assessed accurately.
	