\documentclass[a4paper, 12pt, oneside]{sastra1}
\usepackage{amsmath}
\usepackage{times}
 \usepackage{t1enc}
\usepackage[toc,page]{appendix}
\usepackage{graphicx}
\usepackage{epstopdf}
\usepackage{datetime}
\usepackage[driverfallback=dvipdfm]{hyperref}
%\usepackage[hypertex]{hyperref} % hyperlinks for references.
 % easier math formulae, align, subequations \ldots

\begin{document}
\onehalfspacing
%\begin{document}
	
	%%%%%%%%%%%%%%%%%%%%%%%%%%%%%%%%%%%%%%%%%%%%%%%%%%%%%%%%%%%%%%%%%%%%%%
	%%%% OLD METHOD OF CREATING A FRONT PAGE %%%%%%%%%%%%%%
	\thispagestyle{empty}
	\begin{center}
		\Large{\textbf{PROJECT TITLE}}
	\end{center}
	\bigskip{}
	\bigskip{}
	\bigskip{}
	\begin{center}
		\textit{Thesis submitted to the SASTRA Deemed to be University\\ 
			in partial fulfillment of the requirements\\
			for the award of the degree of\\
		}
		\bigskip{}
		\bigskip{}
		\large{\textbf{M. Tech. POWER \& ENERGY SYSTEMS}}
		\bigskip{}
		\bigskip{}
		\bigskip{}
		\bigskip{}
		\bigskip{}
		\bigskip{}
	\end{center}
	\begin{center}
		\textit{Submitted by}\\
	\end{center}
	\begin{center}
		\begin{singlespacing}
				\textbf{\Large{NAME1}}
			
			\textbf{\large{(Reg. No.: 1234567890)}}
		
		\end{singlespacing}
	\end{center}
	\bigskip{}
	
	\begin{center}
		\Large{\textbf{June 2023}}   %%%%%%%%%% if needed use capital 'L' or small 'l' for large
	\end{center}
	\bigskip{}
	\begin{center}
		\includegraphics[height=1.52in, width=5.65in]{SASTRA_Combined_Logo}
	\end{center}
	
	%\noalign{\smallskip}
	\begin{center}
		\large{\textbf{SCHOOL OF ELECTRICAL \& ELECTRONICS ENGINEERING}} %%%% contents in this line has to be 16
		{\textbf{THANJAVUR,TAMIL NADU, INDIA-613 401}}
	\end{center}
	
	%%%%%%%%%%%%%%%%%%%%%%%%%%%%%%%%%%%%%%%%%%%%%%%%%%%%%%%%%%%%%%%%%%%%%%
	
	\newpage
	
	\pagenumbering{roman}
	\setcounter{page}{2}
	
	\certificate
	
	%\vspace*{0.5in}
	
	\begin{doublespace}
		\linespread{2}
		
		This is to certify that the thesis titled ``\textbf{Project Title (Title Case - First Letter Caps)}'' submitted in partial fulfillment of the requirements for the award of the degree of M. Tech. Power \& Energy Systems to the SASTRA Deemed to be University, is a bona-fide record of the work done by \textbf{Mr./Ms. AB(Reg. No. 12345678)} during the final semester of the academic year 2022-23, in the \textbf{School of Electrical and Electronics Engineering}, under my supervision. This thesis has not formed the basis for the award of any degree, diploma, associateship, fellowship or other similar title to any candidate of any University.
		
	\end{doublespace}
	\vspace*{0.4in}
	
	\noindent\textbf{Signature of Project Supervisor:}~	\\ %%%%%Remove this name while submission
	\\
	\textbf{Name with Affliation\hspace*{19mm}:~\textbf{Dr.~Name}~(Designation~/~EEE~/~SEEE)}	\\
	\\
	\textbf{Date\hspace*{48.25mm}:}~31~/~03~/~2021\\%%%%%%% Add date after paranthesis while printing
	
	\vspace*{0.35in}
	
	\noindent Project \textit{Vivavoce} held on
	
	\vspace*{0.50in}
	\noindent \textbf{Examiner-I} \hspace*{120mm} \textbf{Examiner-II}
	
	%%%%%%%%%%%%%%%%%%%%%%%%%%%%%%%%%%%%%%%%%%%%%%%%%%%%%%%%%%%%%%%%%%%%%%
	
	\declaration
	
	%\vspace*{0.5in}
	
	\begin{doublespace}
		\linespread{2}
		
		I declare that the thesis titled ``\textbf{Project Title (Title Case - First Letter Caps)}'' submitted by me is an original work done by me under the guidance of \textbf{Dr. Xyzzzzzz, Designation, School of Electrical and Electronics Engineering, SASTRA Deemed to be University} during the final semester of the academic year 2022-23, in the \textbf{School of Electrical and Electronics Engineering}. The work is original and wherever I have used materials from other sources, I have given due credit and cited them in the text of the thesis. This thesis has not formed the basis for the award of any degree, diploma, associate-ship, fellowship or other similar title to any candidate of any University.\\
		
	\end{doublespace}
	
	%\end{singlespacing}
	\noindent\textbf{Signature of the candidate(s)	:}	
	
	\noindent\textbf{Name of the candidate(s)\hspace{7mm}		: Name 1}\\
	
	\noindent\textbf{Date\hspace*{43.5mm}					:}~31~/~03~/~2021\\%%%%%%% Add date after paranthesis while printing
	
		\newpage
%%%%%%%%%%%%%%%%%%%%%%%%%%%%%%%%%%%%%%%%%%%%%%%%%%%%%%%%%%%%%%%%%%%%%%
%%%% Comment this part if not done any external project
%	\pagenumbering{roman}
%	\setcounter{page}{2}
	
	\externalcertificate
	
	
	
	
	%%%%%%%%%%%%%%%%%%%%%%%%%%%%%%%%%%%%%%%%%%%%%%%%%%%%%%%%%%%%%%%%%%%%%%
	% Acknowledgements
	\acknowledgements
	
	\hspace*{12pt} We express our gratitude to honourable \textbf{Dr.~S.~Vaidhyasubramaniam}, Vice Chancellor SASTRA University forcm the opportunity of pursuing our engineering in this esteemed in institution and carry out the project work.
	
	\par We thank \textbf{Dr.~R.~Chandramouli}, Registrar, SASTRA University for granting permission and extending the facilities in carrying out this project.
	
	\par We express our sincere thanks and gratitude to \textbf{Dr.~K.~Thenmozhi}, Dean, SEEE and \textbf{Dr.~K.~Vijayarekha}, Associate Dean, EEED/SEEE SASTRA Deemed to be University, for her support in the supporting the accomplishment of this work.
	
	\par We also render our sincere thanks to project coordinator, \textbf{Dr.~Name}, Designation/EEE/SEEE, SASTRA Deemed to be University his involvement and encouragement during this project.
	
	\par We would like to thank our guide, \textbf{Dr.~Name}, Designation/EEE/SEEE, SASTRA Deemed to be University, for his guidance and support, that cumulated to his successful project. His emphasis on making learning an experience allowed us to learn while making mistakes and rectifying them to learn ,not only the scientific concepts behind power systems but also the process of analysing results.
	
	\par We would like to thank our friends who supported us. We would also like to thank the lab assistants for helping us with their practical expertise and for providing the necessary software tools.
	
	\par And finally, we would like to acknowledge the appreciation and support that our parents provided to ensure we faced minimal obstacles throughout the project.
	\pagebreak
	
	%%%%%%%%%%%%%%%%%%%%%%%%%%%%%%%%%%%%%%%%%%%%%%%%%%%%%%%%%%%%%%%%%%%%%%
	\abstract

%\noindent KEYWORDS: \hspace*{0.5em} \parbox[t]{4.4in}{\LaTeX ; Thesis; Style files; Format.}
\begin{doublespace}
	\linespread{2}
	\vspace*{24pt}
	
	\noindent A \LaTeX\ class along with a simple template thesis are provided here.  These can be used to easily write a thesis suitable for submission at IIT-Madras.  The class provides options to format PhD, MS, M.Tech.\ and B.Tech.\ thesis.  It also allows one to write a synopsis using the same class file.  Also provided is a BIB\TeX\ style file that formats all bibliography entries as per the IITM format.
	
	The formatting is as (as far as the author is aware) per the current institute guidelines.
	
\end{doublespace}
\pagebreak
	
	%%%%%%%%%%%%%%%%%%%%%%%%%%%%%%%%%%%%%%%%%%%%%%%%%%%%%%%%%%%%%%%%%
\begin{singlespace}
	\tableofcontents
	\thispagestyle{empty}
	
	
	\listoffigures
	\addcontentsline{toc}{chapter}{LIST OF FIGURES}
	\listoftables
	\addcontentsline{toc}{chapter}{LIST OF TABLES}
\end{singlespace}


%%%%%%%%%%%%%%%%%%%%%%%%%%%%%%%%%%%%%%%%%%%%%%%%%%%%%%%%%%%%%%%%%%%%%%
\abbreviations

\noindent 
\begin{tabbing}
	xxxxxxxxxxx \= xxxxxxxxxxxxxxxxxxxxxxxxxxxxxxxxxxxxxxxxxxxxxxxx \kill
	\textbf{IITM}   \> Indian Institute of Technology, Madras \\
	\textbf{RTFM} \> Read the Fine Manual \\
\end{tabbing}

\pagebreak

%%%%%%%%%%%%%%%%%%%%%%%%%%%%%%%%%%%%%%%%%%%%%%%%%%%%%%%%%%%%%%%%%%%%%%

\chapter*{\centerline{NOTATIONS}}
\addcontentsline{toc}{chapter}{NOTATIONS}

\begin{singlespace}
	\begin{tabbing}
		xxxxxxxxxxx \= xxxxxxxxxxxxxxxxxxxxxxxxxxxxxxxxxxxxxxxxxxxxxxxx \kill
		\textbf{$r$}  \> Radius, $m$ \\
		\textbf{$\alpha$}  \> Angle of thesis in degrees \\
		\textbf{$\beta$}   \> Flight path in degrees \\
	\end{tabbing}
\end{singlespace}

\pagebreak
\clearpage

% The main text will follow from this point so set the page numbering
% to arabic from here on.

\pagenumbering{arabic}

	%%%%%%%%%%%%%%%%%%%%%%%%%%%%%%%%%%%%%%%%%%%%%%%%%%
%%% More chapters can be added if necessary
	\chapter{INTRODUCTION}
	\label{chap:intro}
	
	
	With the increase in Pollution, fuel demand, global warming and many other socio-economic issues, one could say Electric Vehicles (EV) will be the future means of transport. Transportation sector can be tremendously changed by Electric Vehicles as there will be less carbon emissions.
	
	EVs help to conserve non-renewable natural resources. Due
	this the import of natural gasses and the dependency of a nation on petroleum export countries
	will be reduced. When compared to the recurring expenditure on natural gasses, the cost of
	EVs are low. The maintenance of electric motors is less when compared with traditional non-
	electric motors. Electric vehicles have different methods for conversion of fuel to electricity.
	
	
	
	\par Electric bikes, Cars, rickshaws, trucks, etc are some examples of EVs. Most of the trains including metros are already running through electricity all over the world, but there is a problem in charging EVs in comparison with the motor vehicles EVs can't be charged instantly like Petrol/Diesel Vehicles. 
	So here there is a need for an effective charging scheme.
	
	\par But in present times there are difficulties based on how to charge EV s when there is higher
	load demand. The problem arises when there are many vehicles at the same time to charge
	and prioritize them accordingly, there are lesser number of charging stations and the charging
	the time of vehicles differs for different vehicles. Replacement of battery packs can be one such
	solution but that is quite expensive for people to afford. There will be a significant impact on the power grid by charging EVs. In order to manage EVs charging, there is a need for an intelligent
	charging strategy that supports EVs charging while preventing the power grid from
	overloading. Charging/discharging pattern scheme using Real Time Price can be one possible
	solution for the difficulties mentioned above and the same is discussed below.
	
	\par Today when advancement of technology is growing rapidly EVs are the new means of transport to fulfill the larger demand of people growing day by day. One more important factor to consider regarding EVs is the cost of electricity. There are various types of tariffs followed by the government depending on various factors like maximum demand, Time of the load, Type of the load, amount of the energy used, etc. Real time price is one such system where the prices vary hourly and the consumer is charged a different price for each interval, here price solely depends on the demand in the network prevailing during the time. This benefits the consumers as well the government in cutting down their losses. Also, this system could make the load duration curve flatter.


	\par The remainder of the report is organized as follows:
	\begin{itemize}
		\item A literature survey related to the work and the formulated objectives are presented in the chapter \ref{chap:LR}.
		
		\item The chapter \ref{chap:methodology}  provides detailed explanation
		of methodology and the system model comprising of EV.
				
		\item The chapter \ref{chap:mathmodel} deals with the mathematical modelling of the systems used.
		
		\item The chapter  \ref{chap:results} consists of the results obtained for the considered test systems and are discussed and summarized. 
		\item The conclusions of the work is presented in the chapter \ref{chap:conclusion} with a brief description of the future scope of this work.
	\end{itemize}


\chapter{LITERATURE SURVEY}
	\label{chap:LR}
	
	\cite{base} This paper addresses the problem of stability with the grid .The load profile of electric vehicle charging stations (EVCS) is determined  and its impact of EVCS on the voltage profile of the distribution system. The principle of coordinated charging strategy is approached in this project to find the load profile of EVCS constrained by grid-to-vehicle (G2V) and vehicle-to-grid (V2G). 
	
	\cite{soc} This paper focuses on minimization of the losses in the distribution network by  the Distributed Generators (DGs) systems.The undesirable nature is one of the major challenges in maintaining the power between the DG and the grid. The battery energy storage system serves the purpose,supporting the grid by storing the energy in it. Sizing and placement of DG in the distribution system to support the grid and  battery storage placed after the placement of the DGs is done in this work.Particle Swarm Optimization (PSO) is used for this purpose and for placement of the battery energy storage system a genetic algorithm is used in the IEEE networks such as IEEE-33, IEEE-69, IEEE-118 radial node distribution systems.This project incorporates  the operation of the battery in accordance with the state of charge. 
	
	\cite{rtp} The paper focuses on Cost minimization and export maximisation  by optimising  the plug-in hybrid electric vehicle (PHEV) schedule , thereby reducing the import from the grid, subsequently minimising the overall operational cost. The proposed DR proposes to  reduce imported electricity cost in peak hours by shifting the non-emergency loads(controllable loads) to off-peak hours.In this project the per hour Real time pricing(RTP) and load profile has been incorporated and analysed.
	
	\cite{evdata} A dynamic EV charging dataset from the ACN data has been used as the reference to calculate the charging time, discharging time, kWh required, time required to full charge. With this data the charging-discharging pattern has been established. 
	
	\cite{33bus} This paper aims to establish a generalised procedure for evaluating the impact of Electric Vehicles Charging Stations (EVCS) on a distribution system. A load profile for 24 hours of EVCS is proposed using the travel patterns of EV .Radial Load Flow is performed on the distribution system.This is done when the EVCS is connected to the distribution system to find the voltage profile of the system. Voltage Stability Index (VSI) and Voltage Stability Factor (VSF) are calculated for the system with and without EVCS under many cases.
	
	\cite{rao} In this paper, an approach to reconfigure and install DG units simultaneously in the distribution system has been proposed which includes different loss reduction methods to establish the superiority of the proposed method. An effective meta heuristic HSA is used in the process of the network reconfiguration and installation of DG.33- and 69-bus systems at three different load levels vizLight, nominal, and heavy are used for testing the approached as well as other methods.The result was that the approached method was more effective in reducing power loss and improving the voltage profile as compared to other methods. Then this  was studied at different load levels. The results showed that the percentage power loss reduction was improving as the number of DG installation locations were increasing from one to four, but rate of improvement decreased when locations were increasing from one to four at all load levels. However,when the number of DG installation locations were three the ratio of percentage loss reduction to DG size was the highest. The HSA results were compared with the results of genetic algorithm (GA) and refined genetic algorithm (RGA). Final result was that the HSA performance was better than GA and RGA.
	
	\cite{69bus} In this paper, the consumer loss connected to a radial system has been examined by the Quadratic-loss allocation scheme, which is based on branch current flow. Therefore, it is ensured each consumer has been allocated the losses only at branches for which current contributes to.  A heuristic rule and fuzzy multiobjective algorithm in a radial distribution system is derived to solve the network reconfiguration problem. The analysis demonstrates that the network reconfiguration reduces system real-power loss and most consumers will pay less, due to reduction in the loss allocation. However, this may also result in an increase in the loss allocation to a small group of consumers. This can be addressed by modification of tariff structure. It is also observed that network reconfiguration would influence the real-power loss allocation to each consumer.
	
	\cite{zhang2022fast} The large-scale fast charging of EVs in the distribution network creates an issue of uneven temporal and spatial distribution of charging loads and voltage quality deterioration.  This can be addressed by adjustable charging service fees for adjustable charging service fees. The charging guidance strategy will guide the users to charge reasonably, helping to reduce the charging cost and the distribution network’s voltage quality is enhanced substantially.
	The spatial and temporal distribution is forecasted considering the grid constraints of urban traffic road network-distribution and fast charging loads of private cars based on trip chains and Monte Carlo methods is proposed. The load and voltage quality of the distribution network is taken into consideration and a flexible tiered charging service fee model is established.  The trip time, charging service cost and power consumption are considered to construct a user charging location.  The factors that are analysed for a weighted decision model are : voltage of distribution network nodes, user’s charging pattern,  the spatial and temporal distribution of regional charging loads. Other factors which are analysed include, revenue from charging station revenue \& impact on user charging cost under the proposed strategy. 
	The charging guidance strategy has impact on the distribution network, charging station operators and EV users,  apart from changing the spatial and temporal distribution of fast charging loads/
	The method results in the overall voltage deviation to 4.279 and significantly  improves the  voltage quality of the distribution network level. At user level, overall comprehensive charging cost is reduced to 9.28\% motivating to respond and adapt the strategy. 
	At charging station operators level, there has been a slight fall in their revenues \& profit by the adoption of the charging guidance strategy. However, it is envisaged that in future the profit of charging station operators would go up due to the V2G technology development and the increasing use of EV. 
	The approach of participation in grid interaction would be a futuristic research topic.
	
	
	\cite{simolin2022assessing} This paper assesses the influence on EV charging load through the charging profile modelling method.
	Realistic sampling with four commercial EVs are experimented and compared. Also, REDI shopping centre’s charging session data is used for a large charging site evaluation sample.
	It is found that, to avoid modelling errors, linear charging profile is not recommended for controlled charging.  Measurement-based nonlinear charging profiles wage would lead to most accurate modelling 
	However, usage of a simple, but justified, bilinear charging profile model is also recommended for a reasonably accurate result.
	
	Based on the calculations, it is derived that in the constant voltage stage, the charging currents of commercial EVs decrease around 15.3 mA/s per phase on an average. It is further observed that if this value is used to model charging profiles, it will lead to a reasonable low modelling error of 0.64\%–4.39\% for the highest hourly peak power and the charged energy. 
	Therefore, it would help to study EV charging load modelling from the charging site point-of-view, enhancing its accuracy with reduced computational requirements. Also, the paper demonstrates that in the charging site, the battery temperatures do not have a substantial influence on the charging loads and therefore, may not be considered for further related studies.
	The paper also derives that uncontrolled charging can be modelled more accurately when compared to  controlled charging. The different charging profile modelling  accuracy for different control algorithms needs to be investigated further. However, this can be a challenging task because of the limitations on the currently available commercial EVs. For instance, few EVs may not support vehicle-to-grid or  communicate information, such as SoC, to the control system. In addition, real time measurements are to be used as a baseline for the comparisons, otherwise different modelling methods cannot be assessed accurately.
	

	\chapter{METHODOLOGY}
	\label{chap:methodology}
	
	\section{Vehicle Classification}
	
	 Electric vehicles have been classified primarily into four major categories as shown in Figure~\ref{fig:classification} 
	 	
			\begin{figure}[h]
				\centering
				\includegraphics[width=0.7\linewidth]{./Figures/classification}
				\caption{Vehicle Classification}
				\label{fig:classification}
			\end{figure}
		
	The above classification is made by comparing the battery capacity of the vehicles from the data taken with the battery capacity of the similar kind of vehicles in the market.

	\par {Private vehicles are further classified into E-bikes and E-cars with average battery capacity of 400 $Wh$ to 500 $Wh$ and 40 $kWh$ to 100 kWh respectively. Commercial vehicles are classified into E-Truck and E-Bus with an average battery capacity of 100 $kWh$ and 60 to 548 kWh respectively. Emergency vehicles have a battery capacity of around 105 $kWh$ and VIP vehicles have around 90 $kWh$ to 200 $kWh$.
	}
	
	\section{Travel Pattern Establishment}
	
	Travel pattern for three main vehicle subcategories of the above mentioned vehicle categories namely E-car, E-Truck and E-Bus are now taken and travel patterns of the same have been established by using the Battery capacity, Time taken to full charge, Time period of the vehicle when it is connected to the grid , charging rate and discharging rate \cite{evdata}.
	
	Bus: ID – 16034
	This id is chosen as the battery capacity of the bus is considered to be the best in market. The
	battery capacity range of bus is 199.95 KWH. The time taken for 100\% charging of battery
	happens to be 5 hours 20 minutes. It can run for 24 hours after full charging. The charging of
	bus would take place every alternative day as the bus takes 24 hours to complete its trip. The
	bus would have energy of 12.5 KWH on completion of 20 minutes of charging when plugged
	at 0\% Soc. It would have a time period of 5 hours 40 minutes to keep the vehicle idle or to
	discharge.
	
	Car: ID – 13646
	Car ID 13646 is chosen as it is the best in market for all the EV cars in recent times. The
	battery capacity range of car is 54 KWH. The time taken for full charge when plugged at 0\%
	SoC is 1 hour 40 minutes. It can run for 6 hours 30 minutes after full charge. As the vehicle is
	a car its trip for a day can be 2 times and so the above pattern can be repeated for the second
	part of the day. Charge of the vehicle after 20 minutes of charging when plugged at 0\% SoC
	is 10.8 KWH. After the completion of full charge it will have 3 hours 40 minutes for idle/
	discharge period.
	
	Truck: ID – 4428
	Truck ID 4428 is considered as it is successful in market when compared with other trucks.
	The battery capacity range of bus is 99.2 KWH. It requires complete 4 hours to have 100\%
	battery charge or 100\% Soc. It takes 14 hours to fully utilize 99.2 KWH after full charge of
	the battery. As it takes 14 hours to complete its trip it is restricted for one day travel. The
	charge at the end of 20 minutes is 8.267 KWH. After the charging for 4 full hours it will have
	6 hours for the vehicle to discharge or keep the vehicle idle.
	
	\section{Charging/Discharging pattern Establishment}
	
	Generally for the travel pattern to be established, the time scale will be taken for 24 hours
	counting every hour. Here the time scale is considered to be 72 blocks as 1 hour is split into
	three 20 minute block for 24 hours because the classified vehicles have different battery
	capacity range. This split up is done to find the best efficient charging/discharging pattern in
	order to reduce the cost that is paid to the grid for the beneficial of consumers.
	
	\section{Flowchart}
	
	1. With the help of the device data calculate the time period T for which the vehicle is
	connected to the grid.
	
	\noindent 2. Calculate SoC threshold,t for every vehicle.
	
	\noindent 3. The SoC residual,t and SoC consumed,t are determined for the time block t
	\noindent 4. The SoC residual,t is compared with SoC threshold,t
	\begin{itemize}


	 \item If the condition is true then the time must be equal peak demand. If it is equal then
	the vehicle stays idle. If the time is not equal to peak demand then it is then
	checked with the real price algorithm and if it satisfies the algorithm ten the
	vehicle goes to charging, if not the vehicle stays idle.
	
	\item If the condition is not true then time is checked with the peak demand and the
	vehicle is given a priority of either 1 or 0. 1 means the vehicle has to charge and if
	0 then the vehicle discharges when the time is equal to peak demand. When t and
	peak demand are not equal the vehicle stays idle.
	
		\end{itemize}
	
	\begin{figure}
		\centering
		\includegraphics[width=0.9\linewidth]{Figures/Ev_flowchart}
		\caption{}
		\label{fig:evflowchart}
	\end{figure}


	\section{Real Time Price}
	
		\begin{figure}[!h]
			\centering
			\includegraphics[width=0.7\linewidth]{Figures/rtp}
			\caption{}
			\label{fig:rtp}
		\end{figure}
	
	


	\chapter{RESULTS \& DISCUSSION}
	\label{chap:results}
		In this chapter the inclusion of EVs and its impact of power loss in the presence of loads in the system is analyzed.
		

		
		
		
	\section{Simulation Outputs}
	The datas given below are about the Most Economic Time interval to Charge the Vehicle, charging Patterns and Voltage magnitude graphs for different scenarios.
			\subsection{Charging Patterns}
	These plots depict the charging and discharging pattern for the classified vehicles. 216 vehicles are taken into consideration for each vehicle type and the charging pattern is identified.
	
	
	
	\begin{figure}[!h]
		\centering
		\includegraphics[width=0.7\linewidth]{Figures/cp_case1}
		\caption{Charging/Discharging pattern for Cars}
		\label{fig:cpcase1}
	\end{figure}
	
	\begin{figure}[!h]
		\centering
		\includegraphics[width=0.7\linewidth]{Figures/cp_case2}
		\caption{Charging/Discharging pattern for Trucks}
		\label{fig:cpcase2}
	\end{figure}
	
	\begin{figure}[!h]
		\centering
		\includegraphics[width=0.7\linewidth]{Figures/cp_case3}
		\caption{Charging/Discharging pattern for Buses}
		\label{fig:cpcase3}
	\end{figure}
	
	\noindent A strategic pattern is established to maintain the stability of the grid by comparing the vehicle capacity to the Real time price for each hour. When the price is low, the vehicle is put to charge and when the price is high,the vehicle is put to discharge in comparision with the peak demand. By this way the overall cost is drastically reduced , leading to profit for the user and at the same time the overall load in the grid decreases making the load demand curve flatter.
	
	
		\subsection{Economic Charging}
		
			Three cases are taken in this study which are namely:
				\newline Case 1(00:00): Case 1 depicts that the vehicles has Started on a hourly basis i.e. the 1 hour block.
				\newline Case 2(00:20) : Case 2 depicts that the vehicle has Started from the 20th min of an hour i.e. 20mins block.
				\newline Case 3(00:40): Case 3 depicts that the vehicles has Started from the 40th minute of an hour i.e. 40th minute block
			
			\begin{table}[!h]
				\caption{Most Economic Time interval to Charge the Vehicle}
				\centering
			\begin{tabular}{|l|l|l|l|}
				\hline
				\textbf{VEHICLE TYPE}           & \textbf{CASE}         & \textbf{BEST PRICE} & \textbf{HOUR} \\ \hline
				\multirow{3}{*}{\textbf{CAR}}   & \textbf{Case 1 - (00:00)} & - \$1.71            & 12:00         \\ \cline{2-4} 
				& \textbf{Case 2 - (00:20)} & - \$2.09            & 01:20         \\ \cline{2-4} 
				& \textbf{Case 3 - (00:40)} & - \$1.58            & 01:40         \\ \hline
				\multirow{3}{*}{\textbf{BUS}}   & \textbf{Case 1 - (00:00)} & - \$0.47            & 12:00         \\ \cline{2-4} 
				& \textbf{Case 2 - (00:20)} & - \$1.25            & 20:20         \\ \cline{2-4} 
				& \textbf{Case 3 - (00:40)} & - \$0.69            & 11:40         \\ \hline
				\multirow{3}{*}{\textbf{TRUCK}} & \textbf{Case 1 - (00:00)} & - \$2.57            & 05:00         \\ \cline{2-4} 
				& \textbf{Case 2 - (00:20)} & -\$1.00             & 04:20         \\ \cline{2-4} 
				& \textbf{Case 3 - (00:40)} & - \$2.11            & 04:40         \\ \hline
			\end{tabular}
		\label{table:ec}
		\end{table}
		
		

	%\subsection{Maximum EV Load}
	
		In the table \ref{table:whichbus} the hour at which the maximum load is present is shown for Scenario-1 and Scenario-2 along with the load present in the system.
		
			\begin{table}[!h]
			\centering
			\caption{Hour at which the EV Load is maximum }
			\begin{tabular}{|ll|ll|ll|ll|}
				\hline
				\multicolumn{2}{|l|}{\textbf{Scenario}} & \multicolumn{2}{l|}{\textbf{Case-1}} & \multicolumn{2}{l|}{\textbf{Case-2}} & \multicolumn{2}{l|}{\textbf{Case-3}} \\ \hline
				\multicolumn{2}{|l|}{\textbf{Scenario-1}}        & \multicolumn{2}{l|}{$ 19^{th} $}            & \multicolumn{2}{l|}{$ 6^{th} $}             & \multicolumn{2}{l|}{$ 19^{th} $}             \\ \hline
				\multicolumn{2}{|l|}{\textbf{Scenario-2}}        & \multicolumn{2}{l|}{$ 13^{th} $}           & \multicolumn{2}{l|}{$ 6^{th} $}             & \multicolumn{2}{l|}{$ 13^{th} $}             \\ \hline
			\end{tabular}
			\label{table:whichbus}
		\end{table}
	

	
	\section{33 bus system}
		In the 33 bus system the EVs are placed at buses 2 (Strongest Bus) and 18 (Weakest Bus) with Voltage Stability Index as a consideration \cite{33bus} .
 		In the provided table \ref{table:powerloss} the power losses are calculated for the corresponding time blocks  as shown in table \ref{table:whichbus} along with the available loads for that particular hour. The results are tabulated in \ref{table:powerloss}.
 		The voltage magnitude of the base case and each time block for  all the cases are computed. Load flow analysis is done on Bus number 2 and Bus number 18 as these are considered to be the best case and worst case buses to calculate the voltage magnitude as referred from \cite{33bus}. From the results it is evident that the loss difference in the grid without EVs connected to the grid(base case) and with EVs connected to the grid on bus number 2 are comparatively lesser. The same loss difference is higher when the EVs connected in 18th bus is high. Looking into voltage magnitude the value of voltage magnitude in base case and EVs connected to the grid on 2nd bus are equal. Whereas in 18th bus it is clear that voltage magnitude when compared to base case is high.
 		
 		\noindent Here figures (\ref{fig:LFa}), (\ref{fig:LFb}) and  (\ref{fig:LFc}) shows the voltage magnitude plot when the EVs are connected to bus 2 and bus 18 during various cases.
 		
 		\noindent and figures (\ref{fig:LF2a}), (\ref{fig:LF2b}) and  (\ref{fig:LF2c}) shows the voltage magnitude plot when the EVs are connected to bus 2 and bus 18 during various cases.
 		
 		
 		
 		
 		
 		\noindent The base case power loss of the 33 bus system is 202.691 `$kWh$' i.e. when the load demand is 100 percent. The summation of all 72 individual loads of cars, trucks and buses for each case is calculated. These additive loads are identified in each time block and the maximum of it is taken into calculation for load flow.

 		\textbf{Case 1:}
 			\begin{itemize}
 				\item Case one is the hourly connection of vehicles starting at 00:00 and the peak load of EVs is found at $19^{th}$ hour with a load of 284.97 $kW$ 
 				\item This load is connected in bus 2 and bus 18 under two different scenarios with active load demand 1 and 0.94 respectively.
 				\item The resulting power loss is 204.104 $kW$ and 253.746 $kW$ respectively for scenario 1 and 189.833 $kW$ and 236.912 $kW$ respectively for scenario 2.
 				\item This shows a 24.49 percent higher loss in scenario 1 and 10.54 percent higher loss in scenario 2.
 			\end{itemize}

 		\textbf{Case 2:}
 		\begin{itemize}
 			\item Case two is the hourly  connection of vehicles starting at 00:20 and the peak load of EVs is found at $6^{th}$ hour with a load of 276.703 $kW$ 
 			\item This load is connected in bus 2 and bus 18 under two different scenarios with active load demand 0.173913043 and 0.83 respectively.
 			\item The resulting power loss is 65.773 $kW$ and 78.460 $kW$ respectively for scenario 1 and 178.967 $kW$ and 222.468 $kW$ respectively for scenario 2.
 			\item This shows a 6.26 percent higher loss in scenario 1 and 1.943 percent higher loss in scenario 2.
 		\end{itemize}
 		
 		\textbf{Case 3:}
 		\begin{itemize}
 			\item Case three is the hourly connection of vehicles starting at 00:40 and the peak load of EVs is found at $19^{th}$ hour with a load of 284.97 $kW$ 
 			\item This load is connected in bus 2 and bus 18 under two different scenarios with active load demand 1 and 0.94 respectively.
 			\item The resulting power loss is 204.104 $kW$ and 253.746 $kW$ respectively for scenario 1 and 189.833 $kW$ and 236.912 $kW$ respectively for scenario 2.
 			\item This shows a 24.49 percent higher loss in scenario 1 and 10.54 percent higher loss in scenario 2.
 		\end{itemize}
 		
 		From the figure \ref{fig:loadprofile33},  we can identify that the difference in voltage magnitudes differ highly from the base case on the $18^{th}$ bus and remains almost constant on the $2^{nd}$ bus, this concludes that Bus 18 is the weakest bus and Bus 2 isthe strongest bus in this 33 bus system. 
 		
 		
 		
 		
 		
 		
 		
 		
 		
 		
 		
 		
 		
 		
 		
 				\begin{table}[!t]
 			\centering
 			\caption{Power Loss when EV is connected in a 33 bus system}
 			\begin{tabular}{|c|c|cc|cc|cc|}
 				\hline
 				\multirow{2}{*}{\textbf{Scenario}}                            & \multirow{2}{*}{\textbf{\begin{tabular}[c]{@{}c@{}}Base\\ Case\\ Power\\ Loss\\ (kW)\end{tabular}}} & \multicolumn{2}{c|}{\textbf{Case 1}}                                                                                                                                                                          & \multicolumn{2}{c|}{\textbf{Case 2}}                                                                                                                                                                          & \multicolumn{2}{c|}{\textbf{Case 3}}                                                                                                                                                                          \\ \cline{3-8} 
 				&                                                                                              & \multicolumn{1}{c|}{\textbf{\begin{tabular}[c]{@{}c@{}}Power \\ Loss\\ when EV\\ in Bus 2\\ (kW)\end{tabular}}} & \textbf{\begin{tabular}[c]{@{}c@{}}Power \\ Loss\\ when EV\\ in Bus 18\\ (kW)\end{tabular}} & \multicolumn{1}{c|}{\textbf{\begin{tabular}[c]{@{}c@{}}Power \\ Loss\\ when EV\\ in Bus 2\\ (kW)\end{tabular}}} & \textbf{\begin{tabular}[c]{@{}c@{}}Power \\ Loss\\ when EV\\ in Bus 18\\ (kW)\end{tabular}} & \multicolumn{1}{c|}{\textbf{\begin{tabular}[c]{@{}c@{}}Power\\  Loss\\ when EV\\ in Bus 2\\ (kW)\end{tabular}}} & \textbf{\begin{tabular}[c]{@{}c@{}}Power \\ Loss\\ when EV\\ in Bus 18\\ (kW)\end{tabular}} \\ \hline
 				\textbf{\begin{tabular}[c]{@{}c@{}}Scenario\\ 1\end{tabular}} & \multirow{2}{*}{202.691}                                                                     & \multicolumn{1}{c|}{204.104}                                                                                    & 253.746                                                                                     & \multicolumn{1}{c|}{65.773}                                                                                     & 78.460                                                                                      & \multicolumn{1}{c|}{204.104}                                                                                    & 253.746                                                                                     \\ \cline{1-1} \cline{3-8} 
 				\textbf{\begin{tabular}[c]{@{}c@{}}Scenario\\ 2\end{tabular}} &                                                                                              & \multicolumn{1}{c|}{189.833}                                                                                    & 236.912                                                                                     & \multicolumn{1}{c|}{178.967}                                                                                    & 222.468                                                                                     & \multicolumn{1}{c|}{189.833}                                                                                    & 236.912                                                                                     \\ \hline
 			\end{tabular}
 			\label{table:powerloss}
 		\end{table}
 	
 		\begin{figure}[!h]
 			\begin{subfigure}{.5\textwidth}
 				\centering
 				\includegraphics[width=.97\linewidth,height= 4.95cm]{./Figures/sc1_case1}  
 				\caption{Case I}
 				\label{fig:LFa}
 			\end{subfigure}
 			\begin{subfigure}{.5\textwidth}
 				\centering
 				\includegraphics[width=.97\linewidth,height= 4.95cm]{./Figures/sc2_case1}  
 				\caption{Case I}
 				\label{fig:LF2a}
 			\end{subfigure}
 			\begin{subfigure}{.5\textwidth}
 				\centering
 				\includegraphics[width=.97\linewidth,height= 4.95cm]{./Figures/sc1_case2}
 				\caption{Case II}
 				\label{fig:LFb}
 			\end{subfigure}
 			\begin{subfigure}{.5\textwidth}
 				\centering
 				\includegraphics[width=.97\linewidth,height= 4.95cm]{./Figures/sc2_case2}
 				\caption{Case II}
 				\label{fig:LF2b}
 			\end{subfigure}
 			\begin{subfigure}{.5\textwidth}
 				\centering
 				\includegraphics[width=.97\linewidth,height= 4.95cm]{./Figures/sc1_case3}
 				\caption{Case III}
 				\label{fig:LFc}
 			\end{subfigure}
 			\begin{subfigure}{.5\textwidth}
 				\centering
 				\includegraphics[width=.97\linewidth,height= 4.95cm]{./Figures/sc2_case3}
 				\caption{Case III}
 				\label{fig:LF2c}
 			\end{subfigure}
 			\caption{ Voltage magnitude variation for different scenarios on 33 bus system }
 			\label{fig:loadprofile33}
 		\end{figure} 
 		
		
	



	\section{69 bus system}
In the provided table \ref{table:powerloss69} the power loss is calculated in such a way that the time blocks having the maximum load  is considered and then load flow analysis is performed on 2nd bus(best case)and 54th bus (worst case) of 69 bus system to find the variation of load with and without EV .


\begin{table}[!h]
				\centering
	\caption{Power Loss when EV is connected in a 69 bus system}
	\begin{tabular}{|c|c|cc|cc|cc|}
		\hline
		\multirow{2}{*}{\textbf{Scenario}}                            & \multirow{2}{*}{\textbf{\begin{tabular}[c]{@{}c@{}}Base\\ Case\\ Power\\ Loss\\ (kW)\end{tabular}}} & \multicolumn{2}{c|}{\textbf{Case 1}}                                                                                                                                                                          & \multicolumn{2}{c|}{\textbf{Case 2}}                                                                                                                                                                          & \multicolumn{2}{c|}{\textbf{Case 3}}                                                                                                                                                                          \\ \cline{3-8} 
		&                                                                                              & \multicolumn{1}{c|}{\textbf{\begin{tabular}[c]{@{}c@{}}Power \\ Loss\\ when EV\\ in Bus 2\\ (kW)\end{tabular}}} & \textbf{\begin{tabular}[c]{@{}c@{}}Power \\ Loss\\ when EV\\ in Bus 54\\ (kW)\end{tabular}} & \multicolumn{1}{c|}{\textbf{\begin{tabular}[c]{@{}c@{}}Power \\ Loss\\ when EV\\ in Bus 2\\ (kW)\end{tabular}}} & \textbf{\begin{tabular}[c]{@{}c@{}}Power \\ Loss\\ when EV\\ in Bus 54\\ (kW)\end{tabular}} & \multicolumn{1}{c|}{\textbf{\begin{tabular}[c]{@{}c@{}}Power\\  Loss\\ when EV\\ in Bus 2\\ (kW)\end{tabular}}} & \textbf{\begin{tabular}[c]{@{}c@{}}Power \\ Loss\\ when EV\\ in Bus 54\\ (kW)\end{tabular}} \\ \hline
		\textbf{\begin{tabular}[c]{@{}c@{}}Scenario\\ 1\end{tabular}} & \multirow{2}{*}{226.28}                                                                      & \multicolumn{1}{c|}{207.402}                                                                                    & 220.681                                                                                     & \multicolumn{1}{c|}{198.458}                                                                                      & 210.909                                                                                     & \multicolumn{1}{c|}{207.402}                                                                                    & 220.681                                                                                     \\ \cline{1-1} \cline{3-8} 
		\textbf{\begin{tabular}[c]{@{}c@{}}Scenario\\ 2\end{tabular}} &                                                                                              & \multicolumn{1}{c|}{134.758}                                                                                    & 144.014                                                                                     & \multicolumn{1}{c|}{77.01}                                                                                    & 77.1223                                                                                     & \multicolumn{1}{c|}{134.758}                                                                                    & 144.014                                                                                     \\ \hline
	\end{tabular}
	\label{table:powerloss69}
\end{table}


The voltage magnitude of the base case and each time block for  all the cases are computed. Load flow analysis is done on Bus number 2 and Bus number 54 as these are considered to be the best case and worst case buses to calculate the voltage magnitude as referred from \cite{base}. From the results it is evident that the loss difference in the grid without EVs connected to the grid(base case) and with EVs connected to the grid on bus number 2 are comparatively lesser. The same loss difference is higher when the EVs connected in 54th bus is high. Looking into voltage magnitude the value of voltage magnitude in base case and EVs connected to the grid on 2nd bus are equal. Whereas in 18th bus it is clear that voltage magnitude when compared to base case is high.

\noindent Here figures (\ref{fig:RFa}), (\ref{fig:RFb}) and  (\ref{fig:RFc}) shows the voltage magnitude plot when the EVs are connected to bus 2 and bus 18 during various cases.

\noindent and figures (\ref{fig:RF2a}), (\ref{fig:RF2b}) and  (\ref{fig:RF2c}) shows the voltage magnitude plot when the EVs are connected to bus 2 and bus 18 during various cases.

\begin{figure}[!h]
	\begin{subfigure}{.5\textwidth}
		\centering
		\includegraphics[width=.97\linewidth,height= 4.95cm]{./Figures/69_sc1_case1}  
		\caption{Case I}
		\label{fig:RFa}
	\end{subfigure}
	\begin{subfigure}{.5\textwidth}
		\centering
		\includegraphics[width=.97\linewidth,height= 4.95cm]{./Figures/69_sc2_case1}  
		\caption{Case I}
		\label{fig:RF2a}
	\end{subfigure}
	\begin{subfigure}{.5\textwidth}
		\centering
		\includegraphics[width=.97\linewidth,height= 4.95cm]{./Figures/69_sc1_case2}
		\caption{Case II}
		\label{fig:RFb}
	\end{subfigure}
	\begin{subfigure}{.5\textwidth}
		\centering
		\includegraphics[width=.97\linewidth,height= 4.95cm]{./Figures/69_sc2_case2}
		\caption{Case II}
		\label{fig:RF2b}
	\end{subfigure}
	\begin{subfigure}{.5\textwidth}
		\centering
		\includegraphics[width=.97\linewidth,height= 4.95cm]{./Figures/69_sc1_case3}
		\caption{Case III}
		\label{fig:RFc}
	\end{subfigure}
	\begin{subfigure}{.5\textwidth}
		\centering
		\includegraphics[width=.97\linewidth,height= 4.95cm]{./Figures/69_sc2_case3}
		\caption{Case III}
		\label{fig:RF2c}
	\end{subfigure}
	\caption{ Voltage magnitude variation for different scenarios on 69 bus system  }
	\label{fig:loadprofile69}
\end{figure} 
	
	


		
	
	
	

\chapter{CONCLUSIONS AND FURTHER WORK}
	\label{chap:conclusion}
	
		\noindent A \LaTeX\ class along with a simple template thesis are provided here.  These can be used to easily write a thesis suitable for submission at IIT-Madras.  The class provides options to format PhD, MS, M.Tech.\ and B.Tech.\ thesis.  It also allows one to write a synopsis using the same class file.  Also provided is a BIB\TeX\ style file that formats all bibliography entries as per the IITM format.
%%%%%%%%%%%%%%%%%%%%%%%%%%%%%%%%%%%%%%%%%%%%%%%%%%%%%%%%%%%%%%%%%


% Bibliography.
\addcontentsline{toc}{chapter}{REFERENCES}
%\chapter{REFERENCES}
%\label{chap:refs}
%\bibliographystyle{plain}
%\bibliography{mybib}
\begin{singlespace}
	%\section{REFERENCES}
	%\bibliography{MyCollection}.
	\bibliography{refs}
\end{singlespace}
%%%%%%%%%%%%%%%%%%%%%%%%%%%%%%%%%%%%%%%%%%%%%%%%%%%%%%%%%%%%

\listofpapers %%%% If there is no publications remove this section

\begin{enumerate}  
	\item Authors....  \newblock
	Title...
	\newblock {\em Journal}, Volume,
	Page, (year).
\end{enumerate}  


%%%%%%%%%%%%%%%%%%%%%%%%%%%%%%%%%%%%%%%%%%%%%%%%%%%%%%%%%%%%
% Appendices.

\appendix
%\addcontentsline{toc}{chapter}{APPENDICES}
%\begin{appendices}
%
\chapter{FIRST SET DATA}
%
%%Just put in text as you would into any chapter with sections and
%%whatnot.  Thats the end of it.
%
%%%%%%%%%%%%%%%%%%%%%%%%%%%%%%%%%%%%%%%%%%%%%%%%%%%%%%%%%%%%%
%
%
\chapter{SECOND SET DATA}
%
%%Just put in text as you would into any chapter with sections and
%%whatnot.  Thats the end of it.
%
%%%%%%%%%%%%%%%%%%%%%%%%%%%%%%%%%%%%%%%%%%%%%%%%%%%%%%%%%%%%%
%
%
%\chapter{THIRD SET DATA}
%
%%Just put in text as you would into any chapter with sections and
%%whatnot.  Thats the end of it.
%\end{appendices}


%%%%%%%%%%%%%%%%%%%%%%%%%%%%%%%%%%%%%%%%%%%%%%%%%%%%%%%%%%%%


\plagarism
\end{document}
